\documentclass[12pt]{scrartcl}
\usepackage[answers]{probsoln}
\usepackage{amsmath}
\usepackage{amsfonts}
\newtheorem{problem}{Problem}
\usepackage{enumerate}
\usepackage{enumitem}
\newenvironment{myenum}{
\begin{enumerate}[leftmargin=*]
  \setlength{\itemsep}{0.5em}
  \setlength{\parskip}{0pt}
  \setlength{\parsep}{0pt}
}{\end{enumerate}}
\usepackage{graphicx}
\usepackage{color}
\usepackage{pifont}
\usepackage[verbose]{wrapfig}
\usepackage{subfig}
\usepackage{float}
\usepackage{xparse}
\usepackage{environ}
\usepackage{etoolbox}
\PSNrandseed{4} % This randomizes problems every time
\begin{document}
%\toggletrue{doPrintSolution}
%\toggletrue{doPrintMC}
\author{José F. Apolinar}
\title{Preparing exam and homework material with \texttt{probsoln}}
\maketitle

\section*{Purpose}
The following are some thought in using this packages:
\begin{itemize}
  \item When defining problems you don't necessarily need to provide a solution
  \item Solution can be an image of written solution
  \item IMPORTANT? Provide three close answers (or two and one dummy answer) but not necessarily work for them. This will help build shuffle database or any MC assessment.
  \item If taken from a book, article, online, etc. provide citation! This is were superbib comes into play. For example \begin{verbatim}\cite[p.~150]{reference}\end{verbatim}
  \item This list will have the \textbf{absolute} number associated with the problem. Worksheet and assessments will have the \textbf{relative} number i.e. relative to the worksheet, etc.
\end{itemize}


% \doforrandN{1}{\thisfile}{dbase,dbase1,dbasejaja}{%
% \loadrandomproblems{1}{\thisfile}}
% \loadrandomproblems[John]{2}{dbase} % We are calling dbase -> John
%\loadselectedproblems[haha]{2}{dbasejaja} % This loads one problem from the database

% \centerline{\textbf{Homework for John}}
% {\hideanswers
% \begin{myenum}
% \foreachproblem{\item\thisproblem}
% \end{myenum}
% }

% \centerline{\textbf{Homework for John}}
% {\hideanswers
% \begin{myenum}
% \foreachproblem[John]{\item\thisproblem}
% \end{myenum}
% }

% The following selects Problem 1 from dbase named haha:
% {\hideanswers
% \begin{myenum}
% \foreachproblem[haha]{\item\thisproblem}
% \end{myenum}
% }

% %\foreachproblem[haha]{\thisproblem}
% \centerline{\textbf{Solutions for John}}
% {\showanswers
% \begin{myenum}
% \foreachproblem[John]{\color{red}\item\thisproblem}
% \end{myenum}
% }

\section{Probability}
\loadallproblems[test1]{dbase1}

{\hideanswers
\begin{myenum}
  \foreachproblem[test1]{\item\thisproblem}
\end{myenum}
}

\section{Calculus}
\loadallproblems[test2]{dbase2}
{\hideanswers
\begin{myenum}
\foreachproblem[test2]{\item\label{prob:\thisproblemlabel}\thisproblem}
\end{myenum}
}
\section{Solutions for the nice lists of problems}
\showanswers
\textbf{Probability}
\begin{myenum}
\foreachproblem[test1]{\item[\ref{prob:\thisproblemlabel}]\thisproblem}
\end{myenum}

\textbf{Calculus}
\begin{myenum}
\foreachproblem[test2]{\item[\ref{prob:\thisproblemlabel}]\thisproblem}
\end{myenum}
%

\end{document} 